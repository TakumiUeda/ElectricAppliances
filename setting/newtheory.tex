% !TEX root =../CircuitAnalysis1.tex
\usepackage[framed,thmmarks]{ntheorem}

\usepackage{framed}
\usepackage{color}


\theorembodyfont{\normalfont}
\theoremstyle{plain}
\theoremseparator{.}


\newtheorem{Definition}{$BDj5A(B}[section]
\theoremclass{Definition}
\theoremstyle{break}
\newframedtheorem{mydef}[Definition]{$BDj5A(B}

\theorembodyfont{\normalfont}
\theoremstyle{plain}
\theoremseparator{.}

\newtheorem{Theorem}{$BDjM}(B}[section]
\theoremclass{Theorem}
\theoremstyle{break}
\newframedtheorem{mytheorem}[Theorem]{$BDjM}(B}


\newtheorem{Axion}{$B8xM}(B}[section]
\newtheorem{plac}{$B1i=,(B}[section]
\newtheorem{ans}{$B2r(B}[section]
\newtheorem{Proposition}{$BL?Bj(B}[section]
\newtheorem{exam}{$BNc(B}[section]
\newtheorem{Proof}{$BF3=P$d2r@b(B}[section]




%%% real field R
\newcommand{\Rfield}{\mbox{\bf R}}
%%% integer field Z
\newcommand{\Ifield}{\mbox{\bf Z}}
%%% positive integer field Z+
\newcommand{\PIfield}{\mbox{\bf Z$^{+}$}}
%%% natual integer field N
\newcommand{\NIfield}{\mbox{\bf N}}
%%%
\newcommand{\Qfield}{\mbox{\bf Q}}

%%%  grad, div, rot $B5-9f(B
\newcommand{\vgrad}[1]{\mbox{grad }#1}
\newcommand{\vdiv}[1]{\mbox{div }\bm{#1}} 
\newcommand{\vrot}[1]{\mbox{rot }\bm{#1}}
\newcommand{\lhaplus}[1]{\bm{\nabla}^{2} \bm{#1}}
\newcommand{\cross}[2]{\bm{#1} \times \bm{#2}}
\newcommand{\scross}[2]{\bm{a} \times \bm{a}}
\newcommand{\idot}[2]{\bm{#1} \cdot \bm{#2}}


%%%  1/2
\newcommand{\half}{\frac{1}{2}}

   %%%  $B5U%Y%/%H%k%"%/%;%s%H(B
   \newcommand{\ivec}[1]{\stackrel{\leftarrow}{#1}}

   %%%  $BDj5A5-9f(B
   \newcommand{\defeq}{\stackrel{\mbox{\footnotesize def}}{=}}

   %%%  $B5U?t(B
   \newcommand{\rev}[1]{\frac{1}{#1}}

   %%%  $BJPHyJ,(B
   \newcommand{\pdrv}[2]{\dfrac{\partial #1}{\partial #2}}
   \newcommand{\spdrv}[2]{\dfrac{\partial^{2} #1}{\partial^{2} #2}}

   %%%  $BA4HyJ,(B
   \newcommand{\drv}[2]{\dfrac{d#1}{d#2}}

   %%%  $BJQJ,(B
   \newcommand{\ddrv}[2]{\dfrac{\delta #1}{\delta #2}}

   %%%  < $B!J%V%i!K$H(B > ($B%1%C%H(B)
   \newcommand{\lag}{\langle}
   \newcommand{\rag}{\rangle}


   %%%  boldmath $B$G=PNO(B
   \newcommand{\vecbm}[1]{\mbox{\boldmath $#1$}}
